%ustawienia
\documentclass[12pt,a4paper]{article}
\usepackage[T1]{fontenc}
\usepackage{mathptmx}
\usepackage[utf8]{inputenc}
\usepackage{amssymb}
\usepackage[polish]{babel}
\usepackage{polski}
\usepackage{amsmath}
\usepackage{amsfonts}
\usepackage[left=3.5cm,right=2cm,top=2.5cm,bottom=2.5cm]{geometry}
\usepackage{graphicx}
\usepackage{indentfirst} 
\usepackage{float}
\usepackage{hyperref}
\usepackage[most]{tcolorbox}
\setlength{\parindent}{0.7cm}
\hypersetup{
	colorlinks = true,
	linkcolor = black,
	filecolor = magenta,
	urlcolor = blue,
	}
\urlstyle{same}	
	
\author{Karpiński Maciej\\Krysa Marcin\\Kuczma Łukasz\\Mertuszka Adam\\\\\\\includegraphics[width=0.7\linewidth]{img/logoPWSZ.eps}\\\\\\\\Projektowanie i wdrażanie systemów informatycznych}
\title{Nazwa aplikacji}

\begin{document}

	%Stron tytułowa
	\maketitle
	\thispagestyle{empty}
	\pagenumbering{arabic} 
	\clearpage

	%Spis treści
	\tableofcontents
	\newpage

	%Psekcja pierwsza, opisująca firmę
	\section{Prezentacja firmy}
		\indent Opis firmy
	\newpage

	%sekcja druga, Analiza słabych, mocnych, szans i zagrożenia 
	\section{Analiza SWOT}
		\indent Analiza SWOT - Tabela
	\newpage
	
	%Sekcja trzecia, analiza potrzeb
	\section{Analiza potrzeb informacyjnych}
		\indent Analiza potrzeb informacyjnych
	\newpage
	
	%Sekcja czawarta, wybór narzędzi
	\section{Wybór narzędzi}
		\indent Wybór narzędzi skopiować z dokumentacji, dodać ubuntu serwer
		
		Ubuntu 20.04 LTS (Focal Fossa) server jest kompletną dystrybucją systemu operacyjnego GNU/Linux, przeznaczoną do serwerów. Ubuntu bazuje na niestabilnej gałęzi ,,Sid''
		dystrybucji Debian. Projekt rozwijany jest przed przedsiębiorstwo Canonical Ltd. oraz fundację Ubuntu Foundation. Najważniejszymi cechami dystrybucji jest
		\begin{itemize}
			\item Domyślne ustawienia i konfiguracja sprzętu;
			\item Uproszczona administracja;
			\item Bogaty wybór oprogramowania;
			\item Wsparcie techniczne;
		\end{itemize}
		Producent zapewnia wsparcie techniczne i rozwój dystrybucji do kwietnia 2025.
	\newpage
	
	%Sekcja piąta, tworzenie projektu z uwzględnieniem bezpieczeństwa
	\section{Tworzenie projektu z uwzględnieniem bezpieczeństwa}
		\indent Aplikacja została stworzona z uwzględnieniem wszystkich obowiązujących standardów bezpieczeństwa oraz obowiązujących przepisów RODO.
		Przy projektowaniu zabezpieczeń aplikacji pomocna była książka \emph{Bezpieczeństwo aplikacji webowych}\cite{BAW} opisująca liczne metody zabezpieczenia aplikacji
		takich jak szyfrowanie, autoryzacja itp. oraz liczne wpisy opisujące wpadki i ataki na serwisy www i aplikacje webowe opisane na blogach internetowych
		\emph{Niebezpiecznik}\cite{Nieb}, \emph{Sekurak}\cite{Sek} i \emph{Zaufana Trzecia Strona}\cite{ZTS} dzięki, którym udało się wyeliminować dużą ilość potencjalnych wektorów ataków. 
		Przy zabezpieczaniu serwera oraz bazy danych pomocna była książka \emph{Unix i Linux Przewodnik administratora systemów}\cite{Unix} oraz artykuły na blogu internetowym
		\emph{Chris Titus Tech}\cite{CTT} przy pomocy, których udało się zabezpieczyć i zaszyfrować dane znajdujące się na serwerze przed nieautoryzowanym dostępem. Na serwerze
		jest automatycznie tworzona kopia bezpieczeństwa, a dzięki wykorzystaniu macierzy RAID-Z, serwer jest odporny na całkowite uszkodzenie jednego z trzech podłączonych dysków.
		W przeciwieństwie do tradycyjnych macierzy RAID, RAID-Z jest macierzą opartą o system plików ZFS dzięki czemu rozszerzenie macierzy o dodatkowe dyski nie stanowi żadnego problemu
		a przeniesie dysków do innego serwera nie jest uzależnione od obecności identycznego fizycznego kontrolera macierzy RAID.
		\emph{Ogólne rozporządzenie o ochronie danych}\cite{RODO}, \emph{poradniki i wskazówki - UODO}\cite{UODO} wraz z rozmowami z pracownikami wskazały
		zakres danych, które są niezbędne do przetworzenia zlecenia oraz, które w przypadku złamania i/lub ominięcia zabezpieczeń będą jak najmniej
		narażać klientów firmy na wszelkiego rodzaju potencjalne straty.  
	\newpage
	
	%Sekcja szósta, Oszacowanie kosztów - detale
	\section{Oszacowanie kosztów}
		\indent Łączny koszt wdrożenia aplikacji w firmie wyniósł: kwota, na którą składały się poszczególne pozycje:

		\begin{table}[!hbp]
			\center
			\begin{tabular}{|l|c|c|r|}
				\hline
				Nazwa: & Ilość godzin: & Cena za godzinę: & Cena: \\
				\hline
				Analiza zapotrzebowania & \ & \ & \\
				\hline
				Projekt aplikacji & \ & \ & \\
				\hline
				Programowanie frontendu & \ & \ & \\
				\hline
				Programowanie backendu & \ & \ & \\
				\hline
				Programowanie testów & \ & \ & \\
				\hline
				DevOps & \ & \ & \ \\
				\hline
				Poprawki 1 & \ & \ & \ \\
				\hline
				Poprawki 2 & \ & \ & \ \\
				\hline
				Dokumentacja & \ & \ & \ \\
				\hline
				\multicolumn{3}{|l|}{Razem:} & \ \\
				\hline
			\end{tabular}
			\caption{Koszt projektu i programowania aplikacji}
			\label{Koszt projektu i programowania aplikacji}
		\end{table}		
				
		\begin{table}[!hbp]
			\center
			\begin{tabular}{|l|r|}
				\hline
				Dobór podzespołów: & 50zł \\				
				\hline
				Montaż serwera: & 150zł \\
				\hline
				Instalacja i konfiguracja systemu operacyjnego: & 300zł \\
				\hline
				Konfiguracja sieci komputerowej: & 100zł \\
				\hline
				Instalacja i konfiguracji aplikacji: & 100zł \\
				\hline
				Szkolenie pracowników: & 500zł \\
				\hline
				Inne: & 100zł \\
				\hline
				Razem: & 1300zł\\
				\hline
			\end{tabular}
			\caption{Koszt wdrożenia na produkcję}
			\label{Koszt wdrożenie na produkcję}
		\end{table}
		
		\begin{table}[!hbp]
			\center
			\begin{tabular}{|l|l|c|r|}
				\hline
				\multicolumn{2}{|l|}{Nazwa:} & Sztuk: & Cena: \\
				\hline
				Procesor: & AMD Athlon 3000G & 1 & 229zł \\
				\hline
				Płyta główna: & ASRock B450M-HDV R4 & 1 & 259zł \\
				\hline
				Pamięć RAM: & Patriot Viper 4 Blackout & 1 & 179zł \\
				\hline
				Twardy dysk: & Crucial BX 500 & 3 & 687zł \\
				\hline
				Zasilacz: & SilentiumPC Elementum E2 450W & 1 & 150zł \\
				\hline
				Obudowa: & SilentiumPC Armis AR1 & 1 & 109zł \\
				\hline
				\multicolumn{3}{|l|}{Razem:} & 1713zł \\
				\hline
			\end{tabular}
			\caption{Koszt podzespołów serwera}
			\label{Koszt podzespołów serwera}
		\end{table}
	
	\newpage
	
	%Sekcja śiódma, wdrożenie - instalacja systemu
	\section{Wdrożenie}
		\indent Po uprzednim przygotowaniu miejsca w instalację sieci przewodowej Ethernet oraz sieci energetycznej, w którym będzie znajdować się serwer  przystąpiono
		do poskładania wszystkich zakupionych podzespołów serwera a następnie zainstalowania wybranej dystrybucji systemu operacyjnego GNU/Linux. Proces złożenia serwera i instalacji
		systemu operacyjnego, zajęło około dwóch godzin. Następnie przystąpiono do konfiguracji dostępu SSH, stworzenia i skonfigurowania macierzy dyskowej RAID-Z,
		zainstalowania i skonfigurowania firewalla, uruchomienia pełnego szyfrowania dysków. Po wstępnym przygotowaniu serwera został on podłączony w docelowej lokalizacji, a kolejne etapy
		wdrażania zostały przeprowadzone zdalnie poprzez konsolę SSH. Po podłączeniu serwera do sieci Ethernet, został skonfigurowany znajdujący się w przedsiębiorstwie serwer DHCP
		oraz switch. Kolejnymi krokami w wdrażaniu aplikacji było zainstalowanie i skonfigurowanie usługi konteneryzacji Docker oraz pobranie i instalacja stworzonej aplikacji.
		Dzięki modułowej budowie aplikacji, po kolei były uruchamiane i konfigurowane poszczególne elementy systemu aplikacji tj. baza danych MS SQL, Backend-API i Frontend.
		Instalacja zabezpieczenie bazy danych i konfiguracja aplikacja potrwała około jednej godziny. Po zakończeniu instalacji i konfiguracji aplikacji, został skonfigurowany 
		harmonogram systemu ,,cron'' w celu automatycznego uruchamiania skryptu tworzącego kopię bezpieczeństwa a następnie weryfikującego poprawność utworzonej kopii.
		Ze względu na wykorzystanie systemu plików ZFS, nie było potrzeby tworzenia skryptu sprawdzających integralność wcześniej wykonanych kopii zapasowych
		- system plików automatycznie sprawdza i w razie potrzeby dokonuje stosownej korekty. Po zakończeniu konfiguracji wszystkich usług i aplikacji,
		nastąpił etap testowania integralności całego systemu tj. sprawdzono poziomy dostępu do systemu operacyjnego, bazy danych i aplikacji, autoryzację, wszystkie funkcjonalności aplikacji,
		przeprowadzono test jakości kopii zapasowej oraz przeprowadzono symulację uszkodzenia twardego dysku oraz odbudowę macierzy dyskowej,
		sprawdzono konfigurację infrastruktury sieciowej. Na bieżąco były wprowadzane niezbędne poprawki do konfiguracji, a następnie ponownie sprawdzano czy wszystko działa prawidłowo.
		Po zakończeniu testów rozpoczęliśmy szkolenie pracowników z wykorzystania aplikacji.
	\newpage
	
	%Sekcja ósma, Źródła z których były tworzone
	\section{Źródła}
		\begin{thebibliography}{99}
			\bibitem{BAW} M. Bentkowi, G. Coldwind, A. Czyż, R. Janicki, J. Kamiński, A. Michalczyk, M. Niezabitowski, M. Piosek, M. Sajdak, G. Trawiński, B. Widła:
				\emph{Bezpieczeństwo aplikacji webowych},
				SECURITUM Szkolenia sp. z o.o. sp.k., 2019.
				ISBN: 978-83-954853-0-5
			\bibitem{Nieb} https://niebezpiecznik.pl
			\bibitem{Sek} https://sekurak.pl
			\bibitem{ZTS} https://zaufanatrzeciastrona.pl
			\bibitem{Unix} E. Nemeth, G. Snyder, T. R. Hein, B. Whaley, D. Mackin:
				\emph{Unix i Linux Przewodnik administratora systemów Wydanie V},
				Helion S.A., 2018.
				ISBN: 978-83-283-4176-0
			\bibitem{CTT} https://christitus.com
			\bibitem{RODO} https://uodo.gov.pl/pl/404
			\bibitem{UODO} https://uodo.gov.pl/7
		\end{thebibliography}
	\newpage
	
	%Sekcja dziewiąta, spis tabel i obrazków
	\section{Spis tabel i obrazków}
		\listoftables
		\listoffigures
\end{document}