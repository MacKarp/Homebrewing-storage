%ustawienia
\documentclass[12pt,a4paper]{article}
\usepackage[T1]{fontenc}
\usepackage{mathptmx}
\usepackage[utf8]{inputenc}
\usepackage{amssymb}
\usepackage[polish]{babel}
\usepackage{polski}
\usepackage{amsmath}
\usepackage{amsfonts}
\usepackage[left=3.5cm,right=2cm,top=2.5cm,bottom=2.5cm]{geometry}
\usepackage{graphicx}
\usepackage{indentfirst} 
\usepackage{float}
\usepackage{hyperref}
\usepackage[most]{tcolorbox}
\setlength{\parindent}{0.7cm}
\hypersetup{
	colorlinks = true,
	linkcolor = black,
	filecolor = magenta,
	urlcolor = blue,
	}
\urlstyle{same}	
	
\author{Karpiński Maciej\\Krysa Marcin\\Kuczma Łukasz\\Mertuszka Adam\\\\\\\includegraphics[width=0.7\linewidth]{img/logoPWSZ.eps}\\\\\\\\Projektowanie i wdrażanie systemów informatycznych}
\title{Nazwa aplikacji}

\begin{document}

	%Stron tytułowa
	\maketitle
	\thispagestyle{empty}
	\pagenumbering{arabic} 
	\clearpage

	%Spis treści
	\tableofcontents
	\newpage

	%Psekcja pierwsza, opisująca firmę
	\section{Prezentacja firmy}
		\indent Opis firmy
	\newpage

	%sekcja druga, Analiza słabych, mocnych, szans i zagrożenia 
	\section{Analiza SWOT}
		\indent Analiza SWOT - Tabela
	\newpage
	
	%Sekcja trzecia, analiza potrzeb
	\section{Analiza potrzeb informacyjnych}
		\indent Celem projektowanego systemu informatycznego jest usprawnienie realizacji umów z~kontrahentami poprzez skuteczne gromadzenie informacji
			na temat zamówienia, tak aby dział realizujący zamówienie posiadał kompletną listę zamówionych rzeczy przez klienta oraz innych wymaganych do montażu, bez konieczności
			przeglądania umów przez pracownika magazynu w dziale sprzedaży.\\
		\indent System informatyczny składa się z trzech głównych modułów tj:
			\begin{itemize}
				\item Backend - API
				\item Baza danych
				\item Frontend
			\end{itemize}
		\indent Backend - API, realizuje łączność między modułem frontendu a bazą danych, jest odpowiedzialny za przetwarzanie i udostępnianie danych. Backend udostępnia API
			dzięki, któremu	moduł frontendu ma możliwość pobierania i wysyłania niezbędnych danych z bazy danych. Zastosowanie API pozwala w przyszłości rozbudować system o nową
			funkcjonalność lub alternatywne moduły komunikacyjne z użytkownikiem. Budowa backendu uniemożliwia pobieranie zbędnych danych, dzięki czemu ruch sieciowy oraz wymagana moc
			obliczeniowa jest ograniczona do niezbędnego minimum.\\
		\indent Baza danych realizuje funkcję gromadzenia danych wprowadzanych przez użytkowników. Dane wprowadzone do bazy danych są automatycznie archiwizowane
			w kopii zapasowej na twardym dysku zgodnie z ustawionym harmonogramem.\\
		\indent Frontend realizuje funkcje komunikacji pomiędzy użytkownikiem a modułem backendu. Frontend posiada przejrzysty i prosty interfejs użytkownika, przetwarza
			dane pobrane z API w sposób przyjazny dla użytkownika, oraz umożliwia w łatwy sposób wprowadzanie nowych danych do systemu informatycznego.\\			  
		\indent System informacyjny realizuje następujące funkcje:
			\begin{itemize}
				\item Przyjazny interfejs użytkownika - interfejs użytkownika był projektowany i konsultowany wraz z użytkownikami przedsiębiorstwa, którzy będą go codziennie
					wykorzystywać do pracy
				\item Tworzenie i logowanie użytkowników systemu - po otworzeniu okna systemu informatycznego użytkownik będzie miał możliwość zalogowania się na swoje indywidualne konto
					lub w przypadku nowego użytkownika jego utworzenie;
				\item Nadawanie uprawnień użytkownikom - administrator systemu może przypisywać uprawnienia dla kont użytkowników, ograniczając im funkcjonalność jedynie do niezbędnych
					przy wykonywaniu pracy; 
				\item Użytkownik działu sprzedaży może tworzyć i modyfikować ,,kontener'' zawierający listę zamówień, numer umowy oraz termin i adres realizacji zamówienia;
				\item Użytkownik działu magazynowego może przeglądać listę utworzonych ,,kontenerów'' dzięki czemu będzie mógł uzupełnić magazyn o brakujące elementy;
				\item Użytkownik działu realizacji zamówienia może przeglądać listę utworzonych ,,kontenerów'' dzięki czemu wie kiedy i gdzie będzie musiał przystąpić do realizacji zamówienia;
				\item Zabezpieczenie przed utratą danych - serwer został skonfigurowany w taki sposób aby była automatycznie tworzona kopia zapasowa danych, kontrola jej integralności
					oraz możliwe było odtworzenie stanu sprzed awarii maksymalnie jednego twardego dysku. Funkcje realizujące zabezpieczenie danych uruchamiają się automatycznie
					codziennie o godzinie 19:00 oraz powinny zakończyć się maksymalnie o godzinie 5:00;
				\item Zabezpieczenie przed nieautoryzowanym dostępem do danych - dostęp do systemu informatycznego wymaga logowania, wszystkie dane przesyłane pomiędzy systemem informatycznym
					a użytkownikiem są szyfrowane asynchronicznie. Dostęp do serwera wymaga autoryzacji, konto ,,root'' wymaga podania specjalnego hasła, wszystkie dane na dysku
					są automatycznie szyfrowane. Dostęp do bazy danych wymaga autoryzacji, system nie jest dostępny poprzez sieć internetową - działa jedynie w lokalnej sieci Ethernet.
					Zapora sieciowa na serwerze automatycznie blokuje wszelki ruch na portach, które nie są wykorzystywane w realizacji zadań systemu informatycznego, systemu operacyjnego
					oraz usługi konteneryzacji Docker; 
			\end{itemize}
		
	\newpage
	
	%Sekcja czawarta, wybór narzędzi
	\section{Wybór narzędzi}
		\subsection{ASP.NET Core 3.1}
			\indent ASP.Net Core jest wysokowydajnym frameworkiem, do budowania nowoczesnych aplikacji internetowych wykorzystujących moc obliczeniową chmur. ASP.Net Core jest technologią
			open - source, wykorzystującą silnik html Razor, dzięki której możliwe jest tworzenie aplikacji mulitplaformowych, które mogą być używane na każdym urządzeniu wyposażonym
			w przeglądarkę internetową.

		\subsection{AutoMapper}
			\indent AutoMapper jest biblioteką służącą do mapowania między obiektami, dzięki czemu można automatycznie mapować właściwości dwóch różnych obiektów,
					przekształcając obiekt wejściowy jednego typu na obiekt wyjściowy innego typu.  
		\subsection{C\#}
			\indent C\# jest obiektowym językiem programowania, zaprojektowanym w latach 1998 – 2001 dla firmy Microsoft.
			Napisany program jest kompilowany do Common Intermediate Language(CLI), który następnie wykonywany jest w środowisku uruchomieniowym takim jak .NET Framework,
			.NET Core, Mono lub DotGNU.
			Wykorzystanie CLI sprawia, że kod programu jest wieleplatformowy (dopóki istnieje odpowiednie środowisko uruchomieniowe).
			C\# posiada wiele wspólnych cech z językami Object Pascal, Delphi, C++ i Java a najważniejszymi cechami C\# są:
			\begin{itemize}
				\item Obiektowość z hierarchią o jednym elemencie nadrzędnym (podobnie jak w Javie);
				\item Zarządzaniem pamięcią zajmuje się środowisko uruchomieniowe;
				\item Właściwości i indeksery;
				\item Delegaty i zdarzenia – rozwinięcie wskaźników C++;
				\item Typy ogólne, generyczne, częściowe, Nullable, domniemane, anonimowe;
				\item Dynamiczne tworzenie kodu;
				\item Metody anonimowe;
				\item Wyrażenia lambda.
			\end{itemize}
		
		\subsection{Coverlet}
			\indent Coverlet to projekt typu open source, który zapewnia wieloplatformowy framework
			pokrywający kod. Coverlet zbiera dane dotyczące przebiegu testu pokrycia,
			które są używane do generowania raportów.

		\subsection{Docker}
			\indent Docker jest otwarto źródłowym oprogramowaniem służącym do realizacji „konteneryzacji” aplikacji, służąca jako platforma dla programistów
				i administratorów do tworzenia, wdrażania i uruchamiania aplikacji rozproszonych. Pozwala umieścić program oraz jego zależności (biblioteki,
				pliki konfiguracyjne, lokalne bazy danych itp.) w lekkim, przenośnym, wirtualnym kontenerze, który można uruchomić na prawie każdym serwerze
				z systemem Linux. Kontenery wraz z zawartością działają niezależnie od siebie i nie wiedzą o swoim istnieniu. Mogą się jednak ze sobą
				komunikować w ramach ściśle zdefiniowanych kanałów wymiany informacji. Dzięki uruchamianiu na jednym wspólnym systemie operacyjnym,
				konteneryzacja jest lżejszym sposobem wirtualizacji niż pełna wirtualizacja lub parawirtualizacja za pomocą wirtualnych systemów
				operacyjnych.
				 	
		\subsection{Entity Framework}		 
		 	\indent Entity Framework jest technologią open - source do mapowania obiektowo – relacyjnego (ORM), które wspierają rozwój aplikacji zorientowanych na dane.
		 	Entity Framework umożliwia programistom pracę z danymi w postaci obiektów i właściwości specyficznych dla domeny, bez konieczności przejmowania się bazowymi
		 	tabelami i kolumnami baz danych, w których dane są przechowywane. 

		\subsection{Fluent Assertions}
			\indent Fluent Assertions to zestaw metod rozszerzających .NET, które pozwalają
			w bardziej naturalny sposób określić oczekiwany wynik testu jednostkowego.
			Umożliwia to prostą, intuicyjną budowę testu oraz szybsze diagnozowanie przyczyn
			niepowodzenia testu dzięki czytelniejszym błędom.

		\subsection{MailKit}
			\indent MailKit jest multiplatformową otwarto źródłową biblioteką .NET klienta pocztowego opartą o MimeKit, która została zoptymalizowana pod kątem urządzeń mobilnych.
			MailKit oferuje następującą funkcjonalność:
			\begin{itemize}
				\item Obsługa proxy HTTP, Socks4, Socks4a i Socks5;
				\item Uwierzytelnianie SASL;
				\item Kompletny klient SMTP;
				\item Kompletny klient POP3;
				\item Kompletny klient IMAP;
				\item Sortowanie i wątkowanie wiadomości po stronie klienta;
				\item Asynchroniczne wersje wszystkich metod sieciowych;
				\item Obsługa S/MIME, OpenPGP, DKIM i ARC;
				\item Obsługa Microsoft TNEF.
			\end{itemize}

		\subsection{Microsoft SQL Server}		 
		 	\indent Microsoft SQL Server jest systemem zarządzania relacyjnymi bazami danych opracowany przez firmę Microsoft. Cechą charakterystyczną jest głównie wykorzystywanie języka
		 	zapytań	Transact-SQL, który jest rozwinięciem standardu ANSI/ISO. W projekcie wykorzystano wersje 2019 Express, która jest bezpłatną edycją programu Microsoft SQL Server, oferująca
		 	podstawowy silnik bazy danych, nieposiadający ograniczenia ilości obsługiwanych baz lub użytkowników. Ograniczenia, występujące w wersji Express to  m.in.:
		 	korzystanie z~jednego procesora, 1 GB pamięci RAM, 10GB plików bazy danych czy brak SQL Agent.
		
		\subsection{Quartz.NET}
			\indent Quartz.NET jest otwarto źródłową biblioteką do planowania zadań.
			Quartz.NET może być używany do tworzenia prostych lub złożonych harmonogramów wykonywania
			dziesiątek, setek, a nawet dziesiątek tysięcy zadań.
			Quartz.NET jest portem biblioteki Quartz dla środowiska Java. 	 

		\subsection{Swashbuckle}
			\indent Swashbuckle jest biblioteką, która dodaje zestaw narzędzi ,,Swagger" generujących automatycznie dokumentację API aplikacji,
				wyposażoną w przejrzysty interfejs użytkownika. Swashbuckle umożliwia również testowanie API. Dokumentacja jest dostępna pod adresem: ,,/swagger"

		\subsection{Ubuntu 20.04}		
		\indent Ubuntu 20.04 LTS (Focal Fossa) server jest kompletną dystrybucją systemu operacyjnego GNU/Linux, przeznaczoną dla serwerów. Ubuntu bazuje na niestabilnej gałęzi ,,Sid''
			dystrybucji Debian. Projekt rozwijany jest przed przedsiębiorstwo Canonical Ltd. oraz fundację Ubuntu Foundation. Najważniejszymi cechami dystrybucji jest:
			\begin{itemize}
				\item Domyślne ustawienia i konfiguracja sprzętu;
				\item Uproszczona administracja;
				\item Bogaty wybór oprogramowania;
				\item Wsparcie techniczne;
			\end{itemize}
			Producent zapewnia wsparcie techniczne i rozwój dystrybucji do kwietnia 2025.
			
		\subsection{xUnit.net}
			\indent xUnit.net to darmowe narzędzie typu open source służące do testowania jednostkowego
			przeznaczone dla platformy .NET Framework, napisane przez oryginalnego autora NUnit.
			xUnit.net współpracuje z platformami Xamarin, ReSharper, CodeRush i TestDriven.NET.
		
	\newpage
	
	%Sekcja piąta, tworzenie projektu z uwzględnieniem bezpieczeństwa
	\section{Tworzenie projektu z uwzględnieniem bezpieczeństwa}
		\indent Aplikacja została stworzona z uwzględnieniem wszystkich obowiązujących standardów bezpieczeństwa oraz obowiązujących przepisów RODO.
		Przy projektowaniu zabezpieczeń aplikacji pomocna była książka \emph{Bezpieczeństwo aplikacji webowych}\cite{BAW} opisująca liczne mechanizmy i metody zabezpieczenia aplikacji
		takich jak asynchroniczne szyfrowanie, autoryzacja itp. oraz liczne wpisy opisujące wpadki i ataki na serwisy www i aplikacje webowe opisane na blogach internetowych
		\emph{Niebezpiecznik}\cite{Nieb}, \emph{Sekurak}\cite{Sek} i \emph{Zaufana Trzecia Strona}\cite{ZTS} dzięki, którym udało się wyeliminować dużą ilość potencjalnych wektorów ataków. 
		Przy zabezpieczaniu serwera oraz bazy danych pomocna była książka \emph{Unix i Linux Przewodnik administratora systemów}\cite{Unix} oraz artykuły na blogu internetowym
		\emph{Chris Titus Tech}\cite{CTT} przy pomocy, których udało się zabezpieczyć i zaszyfrować dane znajdujące się na serwerze przed nieautoryzowanym dostępem. Na serwerze
		jest automatycznie tworzona kopia bezpieczeństwa, a dzięki wykorzystaniu macierzy RAID-Z, serwer jest odporny na całkowite uszkodzenie jednego z trzech podłączonych twardych dysków.
		W przeciwieństwie do tradycyjnych macierzy RAID, RAID-Z jest macierzą opartą o system plików ZFS dzięki czemu rozszerzenie macierzy o dodatkowe dyski nie stanowi żadnego problemu
		a przeniesie dysków do innego serwera nie jest uzależnione od obecności identycznego fizycznego kontrolera macierzy RAID.
		\emph{Ogólne rozporządzenie o ochronie danych}\cite{RODO}, \emph{poradniki i wskazówki - UODO}\cite{UODO} wraz z rozmowami z pracownikami wskazały
		zakres danych, które są niezbędne do przetworzenia zlecenia oraz, które w przypadku złamania i/lub ominięcia zabezpieczeń będą jak najmniej
		narażać klientów firmy na wszelkiego rodzaju potencjalne straty.  
	\newpage
	
	%Sekcja szósta, Oszacowanie kosztów - detale
	\section{Oszacowanie kosztów}
		\indent Łączny koszt wdrożenia aplikacji w firmie wyniósł: kwota, na którą składały się poszczególne pozycje:

		\begin{table}[!hbp]
			\center
			\begin{tabular}{|l|c|c|r|}
				\hline
				Nazwa: & Ilość godzin: & Cena za godzinę: & Cena: \\
				\hline
				Analiza zapotrzebowania & \ & \ & \\
				\hline
				Projekt aplikacji & \ & \ & \\
				\hline
				Programowanie frontendu & \ & \ & \\
				\hline
				Programowanie backendu & \ & \ & \\
				\hline
				Programowanie testów & \ & \ & \\
				\hline
				DevOps & \ & \ & \ \\
				\hline
				Poprawki 1 & \ & \ & \ \\
				\hline
				Poprawki 2 & \ & \ & \ \\
				\hline
				Dokumentacja & \ & \ & \ \\
				\hline
				\multicolumn{3}{|l|}{Razem:} & \ \\
				\hline
			\end{tabular}
			\caption{Koszt projektu i programowania aplikacji}
			\label{Koszt projektu i programowania aplikacji}
		\end{table}		
				
		\begin{table}[!hbp]
			\center
			\begin{tabular}{|l|r|}
				\hline
				Dobór podzespołów & 50zł \\				
				\hline
				Montaż serwera & 150zł \\
				\hline
				Instalacja i konfiguracja systemu operacyjnego & 300zł \\
				\hline
				Konfiguracja sieci komputerowej & 100zł \\
				\hline
				Instalacja i konfiguracji aplikacji & 100zł \\
				\hline
				Szkolenie pracowników & 500zł \\
				\hline
				Inne & 100zł \\
				\hline
				Razem & 1300zł\\
				\hline
			\end{tabular}
			\caption{Koszt wdrożenia na produkcję}
			\label{Koszt wdrożenie na produkcję}
		\end{table}
		
		\begin{table}[!hbp]
			\center
			\begin{tabular}{|l|l|c|r|}
				\hline
				\multicolumn{2}{|l|}{Nazwa:} & Sztuk: & Cena: \\
				\hline
				Procesor & AMD Athlon 3000G & 1 & 229zł \\
				\hline
				Płyta główna & ASRock B450M-HDV R4 & 1 & 259zł \\
				\hline
				Pamięć RAM & Patriot Viper 4 Blackout & 1 & 179zł \\
				\hline
				Twardy dysk & Crucial BX 500 & 3 & 687zł \\
				\hline
				Zasilacz & SilentiumPC Elementum E2 450W & 1 & 150zł \\
				\hline
				Obudowa & SilentiumPC Armis AR1 & 1 & 109zł \\
				\hline
				\multicolumn{3}{|l|}{Razem} & 1713zł \\
				\hline
			\end{tabular}
			\caption{Koszt podzespołów serwera}
			\label{Koszt podzespołów serwera}
		\end{table}
	
	\newpage
	
	%Sekcja śiódma, wdrożenie - instalacja systemu
	\section{Wdrożenie}
		\indent Po uprzednim przygotowaniu miejsca w instalację sieci przewodowej Ethernet oraz sieci energetycznej, w którym będzie znajdować się serwer  przystąpiono
		do poskładania wszystkich zakupionych podzespołów serwera a następnie zainstalowania wybranej dystrybucji systemu operacyjnego GNU/Linux. Proces złożenia serwera i instalacji
		systemu operacyjnego, zajęło około dwóch godzin. Następnie przystąpiono do konfiguracji dostępu SSH, stworzenia i skonfigurowania macierzy dyskowej RAID-Z,
		zainstalowania i skonfigurowania zapory sieciowej, uruchomienia pełnego szyfrowania dysków. Po wstępnym przygotowaniu serwera został on podłączony w docelowej lokalizacji,
		a kolejne etapy	wdrażania zostały przeprowadzone zdalnie poprzez konsolę SSH. Po podłączeniu serwera do sieci Ethernet, został skonfigurowany znajdujący się w przedsiębiorstwie
		serwer DHCP	oraz switch. Kolejnymi krokami w wdrażaniu aplikacji było zainstalowanie i skonfigurowanie usługi konteneryzacji Docker oraz pobranie i instalacja stworzonej aplikacji.
		Dzięki modułowej budowie aplikacji, po kolei były uruchamiane i konfigurowane poszczególne elementy systemu aplikacji tj. baza danych MS SQL, Backend-API i Frontend.
		Instalacja, zabezpieczenie bazy danych i konfiguracja aplikacji potrwała około jednej godziny. Po zakończeniu instalacji i konfiguracji aplikacji, został skonfigurowany 
		harmonogram systemu \emph{cron} w celu automatycznego uruchamiania skryptu tworzącego kopię bezpieczeństwa a następnie weryfikującego poprawność utworzonej kopii.
		Ze względu na wykorzystanie systemu plików ZFS, nie było potrzeby tworzenia skryptu sprawdzających integralność wcześniej wykonanych kopii zapasowych
		- system plików automatycznie sprawdza integralność wszystkich plików i w razie potrzeby dokonuje stosownej korekty. Po zakończeniu konfiguracji wszystkich usług i aplikacji,
		nastąpił etap testowania integralności całego systemu tj. sprawdzono poziomy dostępu do systemu operacyjnego, bazy danych i aplikacji, autoryzację, wszystkie funkcjonalności aplikacji,
		przeprowadzono test jakości kopii zapasowej oraz przeprowadzono symulację uszkodzenia twardego dysku oraz odbudowę macierzy dyskowej,
		sprawdzono konfigurację infrastruktury sieciowej. W przypadku wykrycia nieprawidłowości od razu były wprowadzane niezbędne poprawki do konfiguracji, a następnie ponownie
		sprawdzano czy wszystko działa prawidłowo. Po zakończeniu testów rozpoczęliśmy szkolenie pracowników z wykorzystania aplikacji.
	\newpage
	
	%Sekcja ósma, Źródła z których były tworzone
	\section{Źródła}
		\begin{thebibliography}{99}
			\bibitem{BAW} M. Bentkowi, G. Coldwind, A. Czyż, R. Janicki, J. Kamiński, A. Michalczyk, M. Niezabitowski, M. Piosek, M. Sajdak, G. Trawiński, B. Widła:
				\emph{Bezpieczeństwo aplikacji webowych},
				SECURITUM Szkolenia sp. z o.o. sp.k., 2019.
				ISBN: 978-83-954853-0-5
			\bibitem{Nieb} https://niebezpiecznik.pl
			\bibitem{Sek} https://sekurak.pl
			\bibitem{ZTS} https://zaufanatrzeciastrona.pl
			\bibitem{Unix} E. Nemeth, G. Snyder, T. R. Hein, B. Whaley, D. Mackin:
				\emph{Unix i Linux Przewodnik administratora systemów Wydanie V},
				Helion S.A., 2018.
				ISBN: 978-83-283-4176-0
			\bibitem{CTT} https://christitus.com
			\bibitem{RODO} https://uodo.gov.pl/pl/404
			\bibitem{UODO} https://uodo.gov.pl/7
		\end{thebibliography}
	\newpage
	
	%Sekcja dziewiąta, spis tabel i obrazków
	\section{Spis tabel i obrazków}
		\listoftables
		\listoffigures
\end{document}