%ustawienia
\documentclass[12pt,a4paper]{article}
\usepackage[T1]{fontenc}
\usepackage{mathptmx}
\usepackage[utf8]{inputenc}
\usepackage{amssymb}
\usepackage[polish]{babel}
\usepackage{polski}
\usepackage{amsmath}
\usepackage{amsfonts}
\usepackage[left=3.5cm,right=2cm,top=2.5cm,bottom=2.5cm]{geometry}
\usepackage{graphicx}
\usepackage{indentfirst} 
\usepackage{float}
\usepackage{hyperref}
\usepackage[most]{tcolorbox}
\setlength{\parindent}{0.7cm}
\hypersetup{
	colorlinks = true,
	linkcolor = black,
	filecolor = magenta,
	urlcolor = blue,
	}
\urlstyle{same}	
	
\author{Karpiński Maciej\\Krysa Marcin\\Kuczma Łukasz\\Mertuszka Adam\\\\\\\includegraphics[width=0.7\linewidth]{img/logoPWSZ.eps}\\\\\\\\Projektowanie i programowanie systemów internetowych II}
\title{Nazwa aplikacji}

\begin{document}

	%Stron tytułowa
	\maketitle
	\thispagestyle{empty}
	\pagenumbering{arabic} 
	\clearpage

	%Spis treści
	\tableofcontents
	\newpage

	%Początek pierwszej sekcji opisującej system
	\section{Opis funkcjonalny systemu}
		\indent Opis funkcjonalny systemu
		\begin{itemize}
			\item funkcja 1
			\item funkcja 2
		\end{itemize}
	\newpage

	%sekcja druga, opis wykorzystanych technologii
	\section{Opis technologiczny}
		\indent Przy tworzeniu projektu ,,'' wykorzystano następujące technologie:

		\subsection{ASP.NET Core 3.1}
			\indent ASP.Net Core jest wysokowydajnym frameworkiem, do budowania nowoczesnych aplikacji internetowych wykorzystujących moc obliczeniową chmur. ASP.Net Core jest technologią
			open - source, wykorzystującą silnik html Razor, dzięki której możliwe jest tworzenie aplikacji mulitplaformowych, które mogą być używane na każdym urządzeniu wyposażonym
			w przeglądarkę internetową.

		\subsection{AutoMapper}
			\indent AutoMapper jest biblioteką służącą do mapowania między obiektami, dzięki czemu można automatycznie mapować właściwości dwóch różnych obiektów,
					przekształcając obiekt wejściowy jednego typu na obiekt wyjściowy innego typu.  
		\subsection{C\#}
			\indent C\# jest obiektowym językiem programowania, zaprojektowanym w latach 1998 – 2001 dla firmy Microsoft.
			Napisany program jest kompilowany do Common Intermediate Language(CLI), który następnie wykonywany jest w środowisku uruchomieniowym takim jak .NET Framework,
			.NET Core, Mono lub DotGNU.
			Wykorzystanie CLI sprawia, że kod programu jest wieleplatformowy (dopóki istnieje odpowiednie środowisko uruchomieniowe).
			C\# posiada wiele wspólnych cech z językami Object Pascal, Delphi, C++ i Java a najważniejszymi cechami C\# są:
			\begin{itemize}
				\item Obiektowość z hierarchią o jednym elemencie nadrzędnym (podobnie jak w Javie);
				\item Zarządzaniem pamięcią zajmuje się środowisko uruchomieniowe;
				\item Właściwości i indeksery;
				\item Delegaty i zdarzenia – rozwinięcie wskaźników C++;
				\item Typy ogólne, generyczne, częściowe, Nullable, domniemane, anonimowe;
				\item Dynamiczne tworzenie kodu;
				\item Metody anonimowe;
				\item Wyrażenia lambda.
			\end{itemize}
		
		\subsection{Docker}
			\indent Docker jest otwarto źródłowym oprogramowaniem służącym do realizacji „konteneryzacji” aplikacji, służąca jako platforma dla programistów
				i administratorów do tworzenia, wdrażania i uruchamiania aplikacji rozproszonych. Pozwala umieścić program oraz jego zależności (biblioteki,
				pliki konfiguracyjne, lokalne bazy danych itp.) w lekkim, przenośnym, wirtualnym kontenerze, który można uruchomić na prawie każdym serwerze
				z systemem Linux. Kontenery wraz z zawartością działają niezależnie od siebie i nie wiedzą o swoim istnieniu. Mogą się jednak ze sobą
				komunikować w ramach ściśle zdefiniowanych kanałów wymiany informacji. Dzięki uruchamianiu na jednym wspólnym systemie operacyjnym,
				konteneryzacja jest lżejszym sposobem wirtualizacji niż pełna wirtualizacja lub parawirtualizacja za pomocą wirtualnych systemów
				operacyjnych.
				 	
		\subsection{Entity Framework}		 
		 	\indent Entity Framework jest technologią open - source do mapowania obiektowo – relacyjnego (ORM), które wspierają rozwój aplikacji zorientowanych na dane.
		 	Entity Framework umożliwia programistom pracę z danymi w postaci obiektów i właściwości specyficznych dla domeny, bez konieczności przejmowania się bazowymi
		 	tabelami i kolumnami baz danych, w których dane są przechowywane. 

		\subsection{Microsoft SQL Server}		 
		 	\indent Microsoft SQL Server jest systemem zarządzania relacyjnymi bazami danych opracowany przez firmę Microsoft. Cechą charakterystyczną jest głównie wykorzystywanie języka
		 	zapytań	Transact-SQL, który jest rozwinięciem standardu ANSI/ISO. W projekcie wykorzystano wersje 2019 Express, która jest bezpłatną edycją programu Microsoft SQL Server, oferująca
		 	podstawowy silnik bazy danych, nieposiadający ograniczenia ilości obsługiwanych baz lub użytkowników. Ograniczenia, występujące w wersji Express to  m.in.:
		 	korzystanie z~jednego procesora, 1 GB pamięci RAM, 10GB plików bazy danych czy brak SQL Agent.

		\subsection{Swashbuckle}
			\indent Swashbuckle jest biblioteką, która dodaje zestaw narzędzi ,,Swagger" generujących automatycznie dokumentację API aplikacji,
				wyposażoną w przejrzysty interfejs użytkownika. Swashbuckle umożliwia również testowanie API. Dokumentacja jest dostępna pod adresem: ,,/swagger"

	\newpage
		
	%Sekcja trzecia, instrukcja lokalnego i zdalnego uruchomienia systemu
	\section{Instrukcja lokalnego i zdalnego uruchomienia systemu}
		\subsection{Lokalne uruchomienie systemu}
			\indent opis lokalnego
			
		\subsection{Zdalne uruchomienie systemu}
			\indent opis zdalnego

	\newpage
	
	%Sekcja czwarta, instrukcja uruchamiania testów oraz opis testowanych funkcjonalności
	\section{Instrukcja uruchamiania testów oraz opis testowanych funkcjonalności}

	\newpage

	\section{Wnioski projektowe}
		\indent wnioski		
\end{document}