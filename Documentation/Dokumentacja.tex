%ustawienia
\documentclass[12pt,a4paper]{article}
\usepackage[T1]{fontenc}
\usepackage{mathptmx}
\usepackage[utf8]{inputenc}
\usepackage{amssymb}
\usepackage[polish]{babel}
\usepackage{polski}
\usepackage{amsmath}
\usepackage{amsfonts}
\usepackage[left=3.5cm,right=2cm,top=2.5cm,bottom=2.5cm]{geometry}
\usepackage{graphicx}
\usepackage{indentfirst} 
\usepackage{float}
\usepackage{hyperref}
\usepackage[most]{tcolorbox}
\setlength{\parindent}{0.7cm}
\hypersetup{
	colorlinks = true,
	linkcolor = black,
	filecolor = magenta,
	urlcolor = blue,
	}
\urlstyle{same}	
	
\author{Karpiński Maciej\\Krysa Marcin\\Kuczma Łukasz\\Mertuszka Adam\\\\\\\includegraphics[width=0.7\linewidth]{img/logoPWSZ.eps}\\\\\\\\Projektowanie i programowanie systemów internetowych II}
\title{Magazynek piwosza}

\begin{document}

	%Stron tytułowa
	\maketitle
	\thispagestyle{empty}
	\pagenumbering{arabic} 
	\clearpage

	%Spis treści
	\tableofcontents
	\newpage

	%Początek pierwszej sekcji opisującej system
	\section{Opis funkcjonalny systemu}
		\indent Projekt ,,Magazynek piwosza'' jest aplikacją internetową przeznaczoną dla pasjonatów warzących piwo w domu, którzy chcieliby posiadać listę posiadanych składników.\\
		\indent ,,Magazynek piwosza'' pozwala zarejestrowanemu i zalogowanemu użytkownikowi na tworzenie swoich wirtualnych ,,schowków'' i przypisywania do nich
		składników wraz z ich datami ważności.\\
		\indent Funkcjonalność aplikacji obejmuje:
		\begin{itemize}
			\item Dostęp do aplikacji poprzez przeglądarkę www (PC, smartphone)
			\item rejestrację i logowanie użytkowników,
			\item dodawanie, modyfikowanie i usuwanie kategorii przedmiotów,
			\item dodawanie, modyfikowanie i usuwanie przedmiotów,
			\item dodawanie, modyfikowanie i usuwanie schowków,
			\item wysyłanie powiadomień o zbliżającym się terminie ważności,
			\item logowanie akcji użytkownika.
		\end{itemize}
	\newpage

	%sekcja druga, opis wykorzystanych technologii
	\section{Opis technologiczny}
		\indent Przy tworzeniu projektu aplikacji ,,Magazynek piwosza'' wykorzystano następujące technologie:

		\subsection{ASP.NET Core 3.1}
			\indent ASP.Net Core jest wysokowydajnym frameworkiem, do budowania nowoczesnych aplikacji internetowych wykorzystujących moc obliczeniową chmur. ASP.Net Core jest technologią
			open - source, wykorzystującą silnik html Razor, dzięki której możliwe jest tworzenie aplikacji mulitplaformowych, które mogą być używane na każdym urządzeniu wyposażonym
			w przeglądarkę internetową.

		\subsection{AutoMapper}
			\indent AutoMapper jest biblioteką służącą do mapowania między obiektami, dzięki czemu można automatycznie mapować właściwości dwóch różnych obiektów,
					przekształcając obiekt wejściowy jednego typu na obiekt wyjściowy innego typu.  
		\subsection{C\#}
			\indent C\# jest obiektowym językiem programowania, zaprojektowanym w latach 1998 – 2001 dla firmy Microsoft.
			Napisany program jest kompilowany do Common Intermediate Language(CLI), który następnie wykonywany jest w środowisku uruchomieniowym takim jak .NET Framework,
			.NET Core, Mono lub DotGNU.
			Wykorzystanie CLI sprawia, że kod programu jest wieleplatformowy (dopóki istnieje odpowiednie środowisko uruchomieniowe).
			C\# posiada wiele wspólnych cech z językami Object Pascal, Delphi, C++ i Java a najważniejszymi cechami C\# są:
			\begin{itemize}
				\item Obiektowość z hierarchią o jednym elemencie nadrzędnym (podobnie jak w Javie);
				\item Zarządzaniem pamięcią zajmuje się środowisko uruchomieniowe;
				\item Właściwości i indeksery;
				\item Delegaty i zdarzenia – rozwinięcie wskaźników C++;
				\item Typy ogólne, generyczne, częściowe, Nullable, domniemane, anonimowe;
				\item Dynamiczne tworzenie kodu;
				\item Metody anonimowe;
				\item Wyrażenia lambda.
			\end{itemize}
		
		\subsection{Coverlet}
			\indent Coverlet to projekt typu open source, który zapewnia wieloplatformowy framework
			pokrywający kod. Coverlet zbiera dane dotyczące przebiegu testu pokrycia,
			które są używane do generowania raportów.
		
		\subsection{CSS}
			\indent Kaskadowe arkusze stylów (ang. Cascading Style Sheets, w skrócie CSS) – język służący do opisu formy prezentacji (wyświetlania) stron WWW.
				CSS został opracowany przez organizację W3C w 1996 r. jako potomek języka DSSSL przeznaczony do używania w połączeniu z SGML-em. Arkusz stylów CSS to lista
				dyrektyw (tzw. reguł) ustalających w jaki sposób ma zostać wyświetlana przez przeglądarkę internetową zawartość wybranego elementu (lub elementów)
				(X)HTML lub XML. Można w ten sposób opisać wszystkie pojęcia odpowiedzialne za prezentację elementów dokumentów internetowych, takie jak rodzina czcionek,
				kolor tekstu, marginesy, odstęp międzywierszowy lub nawet pozycja danego elementu względem innych elementów bądź okna przeglądarki.
				Wykorzystanie arkuszy stylów daje znacznie większe możliwości pozycjonowania elementów na stronie, niż oferuje sam (X)HTML. CSS został stworzony w celu
				odseparowania struktury dokumentu od formy jego prezentacji. Separacja ta zwiększa zakres dostępności witryny, zmniejsza zawiłość dokumentu, ułatwia wprowadzanie
				zmian w strukturze dokumentu. CSS ułatwia także zmiany w renderowaniu strony w zależności od obsługiwanego medium (ekran, palmtop, dokument w druku, czytnik ekranowy).
				Stosowanie zewnętrznych arkuszy CSS daje możliwość zmiany wyglądu wielu stron naraz bez ingerowania w sam kod (X)HTML, ponieważ arkusze mogą być wspólne dla wielu dokumentów. 		
		
		\subsection{Docker}
			\indent Docker jest otwarto źródłowym oprogramowaniem służącym do realizacji „konteneryzacji” aplikacji, służąca jako platforma dla programistów
				i administratorów do tworzenia, wdrażania i uruchamiania aplikacji rozproszonych. Pozwala umieścić program oraz jego zależności (biblioteki,
				pliki konfiguracyjne, lokalne bazy danych itp.) w lekkim, przenośnym, wirtualnym kontenerze, który można uruchomić na prawie każdym serwerze
				z systemem Linux. Kontenery wraz z zawartością działają niezależnie od siebie i nie wiedzą o swoim istnieniu. Mogą się jednak ze sobą
				komunikować w ramach ściśle zdefiniowanych kanałów wymiany informacji. Dzięki uruchamianiu na jednym wspólnym systemie operacyjnym,
				konteneryzacja jest lżejszym sposobem wirtualizacji niż pełna wirtualizacja lub parawirtualizacja za pomocą wirtualnych systemów
				operacyjnych.
				 	
		\subsection{Entity Framework}		 
		 	\indent Entity Framework jest technologią open - source do mapowania obiektowo – relacyjnego (ORM), które wspierają rozwój aplikacji zorientowanych na dane.
		 	Entity Framework umożliwia programistom pracę z danymi w postaci obiektów i właściwości specyficznych dla domeny, bez konieczności przejmowania się bazowymi
		 	tabelami i kolumnami baz danych, w których dane są przechowywane. 

		\subsection{Fluent Assertions}
			\indent Fluent Assertions to zestaw metod rozszerzających .NET, które pozwalają
			w bardziej naturalny sposób określić oczekiwany wynik testu jednostkowego.
			Umożliwia to prostą, intuicyjną budowę testu oraz szybsze diagnozowanie przyczyn
			niepowodzenia testu dzięki czytelniejszym błędom.
		\subsection{HTML}
			\indent HTML (ang. HyperText Markup Language) – hipertekstowy język znaczników, wykorzystywany do tworzenia dokumentów hipertekstowych. HTML pozwala opisać
				strukturę informacji zawartych wewnątrz strony internetowej, nadając odpowiednie znaczenie semantyczne poszczególnym fragmentom tekstu – formując hiperłącza,
				akapity, nagłówki, listy – oraz osadza w tekście dokumentu obiekty plikowe np. multimedia bądź elementy baz danych np. interaktywne formularze danych.
				HTML umożliwia określenie wyglądu dokumentu w przeglądarce internetowej. Do szczegółowego opisu formatowania akapitów, nagłówków, użytych czcionek i kolorów,
				zalecane jest wykorzystywanie kaskadowych arkuszy stylów.
				 
		\subsection{JavaScript}
			\indent JavaScript (w skrócie JS) – skryptowy język programowania, stworzony przez firmę Netscape Najczęściej spotykanym zastosowaniem języka JavaScript
				są strony internetowe. Skrypty te służą najczęściej do zapewnienia interakcji poprzez reagowanie na zdarzenia, walidacji danych wprowadzanych w formularzach
				lub tworzenia złożonych efektów wizualnych. Skrypty JavaScriptu uruchamiane przez strony internetowe mają znacznie ograniczony dostęp do komputera użytkownika.
				Po stronie serwera JavaScript może działać w postaci node.js lub Ringo. W języku JavaScript można także pisać pełnoprawne aplikacje. Fundacja Mozilla udostępnia
				środowisko złożone z technologii takich jak XUL, XBL, XPCOM oraz JSLib. Umożliwiają one tworzenie korzystających z zasobów systemowych aplikacji
				o graficznym interfejsie użytkownika dopasowującym się do danej platformy. Przykładem aplikacji napisanych z użyciem JS i XUL może być klient IRC o nazwie
				ChatZilla, domyślnie dołączony do pakietu Mozilla. Microsoft udostępnia biblioteki umożliwiające tworzenie aplikacji JScript jako część środowiska Windows
				Scripting Host. Ponadto JScript.NET jest jednym z podstawowych języków środowiska .NET. Istnieje także stworzone przez IBM środowisko SashXB dla systemu Linux,
				które umożliwia tworzenie w języku JavaScript aplikacji korzystających z GTK+, GNOME i OpenLDAP. Platforma Node.js umożliwia pisanie aplikacji wiersza poleceń oraz
				aplikacji serwerowych. Node.js używany jest także w środowisku Electron, który umożliwia pisanie aplikacji GUI. Język JavaScript używany jest także na urządzeniach
				internetu rzeczy, robotów czy układów takich jak Arduino poprzez bibliotekę Johnny-Five. 		
		
		\subsection{MailKit}
			\indent MailKit jest multiplatformową otwarto źródłową biblioteką .NET klienta pocztowego opartą o MimeKit, która została zoptymalizowana pod kątem urządzeń mobilnych.
			MailKit oferuje następującą funkcjonalność:
			\begin{itemize}
				\item Obsługa proxy HTTP, Socks4, Socks4a i Socks5;
				\item Uwierzytelnianie SASL;
				\item Kompletny klient SMTP;
				\item Kompletny klient POP3;
				\item Kompletny klient IMAP;
				\item Sortowanie i wątkowanie wiadomości po stronie klienta;
				\item Asynchroniczne wersje wszystkich metod sieciowych;
				\item Obsługa S/MIME, OpenPGP, DKIM i ARC;
				\item Obsługa Microsoft TNEF.
			\end{itemize}

		\subsection{Microsoft SQL Server}		 
		 	\indent Microsoft SQL Server jest systemem zarządzania relacyjnymi bazami danych opracowany przez firmę Microsoft. Cechą charakterystyczną jest głównie wykorzystywanie języka
		 	zapytań	Transact-SQL, który jest rozwinięciem standardu ANSI/ISO. W projekcie wykorzystano wersje 2019 Express, która jest bezpłatną edycją programu Microsoft SQL Server, oferująca
		 	podstawowy silnik bazy danych, nieposiadający ograniczenia ilości obsługiwanych baz lub użytkowników. Ograniczenia, występujące w wersji Express to  m.in.:
		 	korzystanie z~jednego procesora, 1 GB pamięci RAM, 10GB plików bazy danych czy brak SQL Agent.
		
		\subsection{Node.js}
		\indent Node.js – wieloplatformowe środowisko uruchomieniowe o otwartym kodzie do tworzenia aplikacji typu server-side napisanych w języku JavaScript.
			Przyczynił się do stworzenia paradygmatu ,,JavaScript everywhere'' umożliwiając programistom tworzenie aplikacji w~obrębie jednego języka programowania zamiast
			polegania na odrębnych po stronie serwerowej. Node.js składa się z silnika V8 (stworzonego przez Google), biblioteki libUV oraz kilku innych bibliotek.
			Domyślnym managerem pakietów dla Node.js jest Npm. 
		
		\subsection{Quartz.NET}
			\indent Quartz.NET jest otwarto źródłową biblioteką do planowania zadań.
				Quartz.NET może być używany do tworzenia prostych lub złożonych harmonogramów wykonywania
				dziesiątek, setek, a nawet dziesiątek tysięcy zadań.
				Quartz.NET jest portem biblioteki Quartz dla środowiska Java. 	 
		
		\subsection{React.js}
			\indent React.js – biblioteka języka programowania JavaScript, która wykorzystywana jest do tworzenia interfejsów graficznych aplikacji internetowych.
				Często wykorzystywana do tworzenia aplikacji typu Single Page Application. Z głównych cech wyróżniających bibliotekę React.js jest wirtualny DOM
				(Document Object Model). React przechowuje cały DOM aplikacji w pamięci, po zmianie stanu wyszukuje różnice między wirtualnym i prawdziwym DOM
				i aktualizuje zmiany. Drugą z cech szczególnych React jest język JSX. Jest on nakładką na JavaScript, która dodaje możliwość wstawiania kodu html
				(lub komponentów React) bezpośrednio w kodzie, zamiast ciągu znaków. React.js jest obecnie używany na stronach internetowych firm takich
				jak Netflix, Imgur, PayPal, Archive.org, Gamepedia, SeatGeek, HelloSign czy Walmart.
		
		\subsection{Swashbuckle}
			\indent Swashbuckle jest biblioteką, która dodaje zestaw narzędzi ,,Swagger" generujących automatycznie dokumentację API aplikacji,
				wyposażoną w przejrzysty interfejs użytkownika. Swashbuckle umożliwia również testowanie API. Dokumentacja jest dostępna pod adresem internetowym:
			\begin{tcolorbox}[minipage,colback=white,arc=0pt,outer arc=0pt, fontupper=\scriptsize]
				\center					
				\url{http://homebrew.tplinkdns.com:81/swagger}
			\end{tcolorbox}

		\subsection{xUnit.net}
			\indent xUnit.net to darmowe narzędzie typu open source służące do testowania jednostkowego
			przeznaczone dla platformy .NET Framework, napisane przez oryginalnego autora NUnit.
			xUnit.net współpracuje z platformami Xamarin, ReSharper, CodeRush i TestDriven.NET.
	
	\newpage
		
	%Sekcja trzecia, instrukcja lokalnego i zdalnego uruchomienia systemu
	\section{Instrukcja lokalnego i zdalnego uruchomienia systemu}
		\subsection{Lokalne uruchomienie systemu}
			\indent Aplikacja ,,Magazynek piwosza'' jest systemem internetowym, który jest dostępny
			poprzez dowolną współczesną przeglądarkę internetową. Aby skorzystać z aplikacji należy
			uruchomić przeglądarkę internetową a następnie udać się pod adres internetowy:
				\begin{tcolorbox}[minipage,colback=white,arc=0pt,outer arc=0pt, fontupper=\scriptsize]
					\center					
					\url{http://homebrew.tplinkdns.com}
				\end{tcolorbox}
			
		\subsection{Zdalne uruchomienie systemu}
			\indent Do uruchomienia systemu wymagana jest działająca platforma konteneryzacji ,,Docker''
			oraz narzędzie ,,Docker Compose'' szczegóły instalacji i konfiguracji wymaganych
			komponentów, można sprawdzić na stronie internetowej:
			\begin{tcolorbox}[minipage,colback=white,arc=0pt,outer arc=0pt, fontupper=\scriptsize]
				\center					
				\url{https://docs.docker.com/}
			\end{tcolorbox}			
			Po poprawnym zainstalowaniu i skonfigurowaniu narzędzi należy utworzyć plik o nazwie: ,,docker-compose.yaml'',
			z następującą przykładową konfiguracją:  
			\begin{tcolorbox}[minipage,colback=white,arc=0pt,outer arc=0pt, fontupper=\footnotesize]
				\begin{tabbing}
						version: '3' \\
						serv\= ices: \\
						\> back\= endapi: \\
        				\>\> image: karpiu/homebrewing-storage-api \\
        				\>\> restart: always \\
        				\>\> container\_name: Backend\_API \\
        				\>\> depe\= nds\_on: \\
            			\>\>\> - db \\ 
        				\>\> environment: \\ 
            			\>\>\> DBServer: Database \\ 
            			\>\>\> DBPort: 1433 \\ 
            			\>\>\> DBUser: SA \\
            			\>\>\> DBPassword: ThisIsNotSuperSecretP@55w0rd \\
            			\>\>\> Database: Backend\_API \\
            			\>\>\> SmtpServer: your.email.smtp.server.here \\
            			\>\>\> SmtpPort: 465 \\
						\>\>\> SSL: "True" \\
            			\>\>\> SmtpUserName: your.email@addres.here \\
            			\>\>\> SmtpUserPassword: ThisIsNotSuperSecretP@55w0rd \\
            			\>\>\> NotificationSchedule: "0 0 3 * * ?" \\
        				\>\> ports: \\
            			\>\>\> - "8080:80" \\
    					\> db: \\
						\>\> image: mcr.microsoft.com/mssql/server:2019-latest \\
        				\>\> restart: always \\
        				\>\> environment: \\
            			\>\>\> ACCEPT\_EULA: Y \\
            			\>\>\> SA\_PASSWORD: ThisIsNotSuperSecretP@55w0rd \\
            			\>\>\> MSSQL\_PID: Express \\
        				\>\> container\_name: Database \\
        				\>\> ports: \\
						\>\>\> - "14331:1433"
					\end{tabbing}
				\end{tcolorbox}	
				\begin{tcolorbox}[minipage,colback=white,arc=0pt,outer arc=0pt, fontupper=\scriptsize]
					Uwaga! Plik konfiguracyjny wymaga odpowiedniego formatowania, przykładowe pliki konfiguracyjne są dostępne w repozytorium projektu:
					\url{https://github.com/MacKarp/Homebrewing-storage}
				\end{tcolorbox}
				Po zapisaniu pliku konfiguracyjnego należy uruchomić następujące konsolowe polecenie w folderze zawierającym plik ,,docker-compose.yaml'':
				\begin{tcolorbox}[minipage,colback=white,arc=0pt,outer arc=0pt, fontupper=\scriptsize]
					docker-compose up
				\end{tcolorbox}
				Spowoduje to pobranie i uruchomienie wszystkich wszystkich składników aplikacji. Aby zakończyć działanie aplikacji należy uruchomić konsolowe
				polecenie w katalogu zawierającym plik ,,docker-compose.yaml''  	
				\begin{tcolorbox}[minipage,colback=white,arc=0pt,outer arc=0pt, fontupper=\scriptsize]
					docker-compose stop			
				\end{tcolorbox}

	\newpage
	
	%Sekcja czwarta, instrukcja uruchamiania testów oraz opis testowanych funkcjonalności
	\section{Instrukcja uruchamiania testów oraz opis testowanych funkcjonalności}
		\indent Wszystkie testy projektowanej aplikacji są automatycznie uruchamiane na platformie GitHub, przy każdym ,,commitcie'' i utworzeniu ,,pull requesta''.
			Lokalnie testy można uruchomić poprzez interfejs graficzny oferujący testy np. Eksplorator testów w Visual Studio 2019 lub poprzez uruchomienie komendy konsolowej:
				\begin{tcolorbox}[minipage,colback=white,arc=0pt,outer arc=0pt, fontupper=\scriptsize]
					dotnet test --verbosity normal
				\end{tcolorbox}
				\begin{tcolorbox}[minipage,colback=white,arc=0pt,outer arc=0pt, fontupper=\scriptsize]
					Uwaga! Aby przeprowadzić lokalne testy należy posiadać uruchomiony lokalnie MSSQL Server na porcie: 14331
				\end{tcolorbox}
				 54 testy integracyjne obejmują poprawne działanie wszystkich endpointów API tj. pobieranie, tworzenie, modyfikację i usuwanie danych.     
	\newpage
	
	\section{Repozytorium kodu i dokumentacja techniczna}
		\indent Repozytorium kodu projektu i jego dokumentacja znajdują się w serwisie GitHub pod adresem: 
			\begin{tcolorbox}[minipage,colback=white,arc=0pt,outer arc=0pt, fontupper=\scriptsize]
				\center									
				\url{https://github.com/MacKarp/Homebrewing-storage}
			\end{tcolorbox}
		\indent Dokumentacja API dostępna jest pod adresem:
				\begin{tcolorbox}[minipage,colback=white,arc=0pt,outer arc=0pt, fontupper=\scriptsize]
					\center					
					\url{http://homebrew.tplinkdns.com:81/swagger}
				\end{tcolorbox}
	\newpage
	
	\section{Wnioski projektowe}
		\indent Projekt aplikacji internetowej było bardzo ciekawym doświadczaniem, w którym zdobyliśmy wiele praktycznej wiedzy jak w poprawny sposób
			stworzyć profesjonalną aplikację internetową począwszy od wyboru wykorzystywanej technologi, poprzez projekt frontendu i oddzielonej funkcjonalności backendu.
			Wdrożenie aplikacji internetowej opartej o framework ASP.NET Core 3.1 okazało się bardzo proste i powtarzalne dzięki zastosowaniu konteneryzacji ,,Dockera''.
			Praca w grupie wykazała że każdy z nas posiada inny zestaw umiejętności i wiedzy dzięki czemu w naturalny sposób wytworzył się podział prac przy projekcie,
			jednocześnie gdy napotykany był jakiś problem to różnice w~posiadanej wiedzy pozwalały na wspólne szybkie rozwiązanie problemu. 
\end{document}