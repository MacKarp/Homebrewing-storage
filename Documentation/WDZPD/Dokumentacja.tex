%ustawienia
\documentclass[12pt,a4paper]{article}
\usepackage[T1]{fontenc}
\usepackage{mathptmx}
\usepackage[utf8]{inputenc}
\usepackage{amssymb}
\usepackage[polish]{babel}
\usepackage{polski}
\usepackage{amsmath}
\usepackage{amsfonts}
\usepackage[left=3.5cm,right=2cm,top=2.5cm,bottom=2.5cm]{geometry}
\usepackage{graphicx}
\usepackage{indentfirst} 
\usepackage{float}
\usepackage{hyperref}
\usepackage[most]{tcolorbox}
\usepackage{fancyhdr}
\setlength{\parindent}{0.7cm}
\hypersetup{
	colorlinks = true,
	linkcolor = black,
	filecolor = magenta,
	urlcolor = blue,
	}
\urlstyle{same}	
	
\author{
	\\\includegraphics[width=0.7\linewidth]{img/logoPWSZ.eps} \\\\\\\\
	\hfill Karpiński Maciej\\
	\hfill Krysa Marcin\\
	\hfill Kuczma Łukasz\\
	\hfill Mertuszka Adam\\\\
	\hfill Prowadzący mgr inż. Marcin Tracz
	}
\title{\textbf{Wprowadzenie do zarządzania projektami deweloperskimi}\\\line(1,0){400}\\\textbf{laboratorium}}
\date{}

\begin{document}

	%Stron tytułowa
	\maketitle
	\thispagestyle{fancy}
	\fancyhf{}
	\rhead{\textcolor{gray}{\footnotesize Państwowa Wyższa Szkoła Zawodowa im. Witelona w Legnicy\\Informatyka, rok III\\Semestr zimowy 2020/2021}}	
	\renewcommand{\headrulewidth}{0pt}
	\clearpage

	%Spis treści
	\pagestyle{fancy}
	\rfoot{\thepage}	
	\tableofcontents
	\newpage

	%opis projektu
	\section{Opis projektu}
		\indent Projekt nazwaProjektu jest aplikacją internetową przeznaczoną dla pasjonatów warzących piwo w domu, którzy chcieliby posiadać listę posiadanych składników
		wraz z terminami ważności.    
	\newpage
	
	%Opis poszczególnych ról w grupie, kto za co był odpowiedzialny
	\section{Role w grupie}
	\newpage
	
	%Opis wymagań i założeń projektowych
	\section{Wymagania i założenia projektowe}
		Wymagania projektowe:
		\begin{itemize}
			\item Uwierzytelnianie (w tym poprzez social media);
			\item Przechowywanie różnych typów produktów;
			\item Powiadomienia;
			\item Poręczny interfejs;
			\item Separacja backendu i frontendu;
			\item Logi.
		\end{itemize}
		Założenia:
		\begin{itemize}
			\item Użytkownik może tworzyć swoje własne schowki;
			\item Użytkownik wybiera składnik z listy składników i dodaje go do swojego schowka;
			\item Użytkownik wprowadza termin ważności składnika;
			\item Użytkownik dostaje powiadomienie o zbliżającym się terminie ważności;
			\item Przyjazny interfejs użytkownika;
		\end{itemize}				 
	\newpage
	
	%Opis działania aplikacji / systemu
	\section{Opis działania}
	\newpage
	
	%Opis wykorzystanych technologii, narzędzi i rozwiązań technicznych
	\section{Wykorzystana technologia i narzędzia}
	\indent skopiować finalny
	\newpage

	%Opis event stormingu + zrzuty z modelowania (wyraźne)
	\section{Event storming}
		\indent Event Storming został przeprowadzony na portalu ,,Miro'' \url{https://miro.com/}, w dniu 05.12.2020 było to nowe i ciekawe doświadczenie dla każdego z członków zespołu.
		Projekt w momencie przeprowadzenia event stormingu był już w zaawansowanej formie, wraz z rozpisanymi zadaniami więc wykorzystaliśmy spotkanie
		do stworzenia modelu pracy aplikacji. Podczas tworzenia wizualizacji okazało się, że rozumienie aspektów działania i reagowania aplikacji na niektóre zdarzenia
		różnią się między poszczególnymi członkami zespołu, ujawnione różnice w wyobrażeniu na temat funkcjonowania poszczególnych komponentów aplikacji udało się wybrać
		najlepsze rozwiązania. Podczas spotkania wyraźnie było widać brak doświadczenia w tego typu zadaniu, na początku często przerywaliśmy sobie wypowiedzi ale dość szybko
		doszliśmy do porozumienia i zapanował porządek. Wspólnie doszliśmy do wniosku że było to ciekawe doświadczenie, chodź lepszym byłoby spotkanie ,,twarzą w twarz''
		wraz z osobą doświadczoną, która mogłaby poprowadzić takie spotkanie. W miarę rozwoju aplikacji do modelu aplikacji były dodawane kolejne szczegóły. 
		Wynikiem event stormingu jest spójny model działania aplikacji, dzięki czemu każdy członek zespołu może prześledzić przepływ informacji.  
	\newpage
	
	%Backlog
	\section{Backlog}
	\newpage
	
	%Estymata
	\section{Estymata}
	\newpage

	%Opis sprintów + zrzuty
	\section{Sprinty}
	\newpage
	
	%Interfejs aplikacji (zrzuty) + krótki opis
	\section{Interfejs aplikacji}
	\newpage
	
	%Informacje uruchomieniowe w środowisku developerskim (instrukcja, co zrobić aby po ściągnięciu projektu z repozytorium GIT, móc uruchomić go na lokalnej maszynie)
	\section{Uruchomienie w środowisku developerskim}
		\indent Repozytorium kodu projektu i jego dokumentacja znajdują się w serwisie ,,GitHub'' pod adresem: 
			\begin{tcolorbox}[minipage,colback=white,arc=0pt,outer arc=0pt, fontupper=\scriptsize]
				\url{https://github.com/MacKarp/Homebrewing-storage}
			\end{tcolorbox}
		\indent Do uruchomienia systemu wymagana jest działająca platforma konteneryzacji ,,Docker''
			oraz narzędzie ,,Docker Compose'' szczegóły instalacji i konfiguracji wymaganych
			komponentów, można sprawdzić na stronie internetowej: \url{https://docs.docker.com/}.
		\indent Po ściągnięciu repozytorium w katalogu głównym projektu znajdują się pliki wsadowe, umożliwiające kompilację i uruchomienie ,,frontendu'' i ,,backendu''
		wraz z bazą danych.
		Pliki wsadowe dla systemu Windows:
		\begin{itemize}
			\item Backend-Start.bat
			\item Backend-Start-DEV.bat
			\item Backend-Stop.bat
			\item Backend-Stop-DEV.bat
		\end{itemize}
		Pliki dla systemu Linux:
		\begin{itemize}
			\item Backend-Start.sh
			\item Backend-Start-DEV.sh
			\item Backend-Stop.sh
			\item Backend-Stop-DEV.sh
		\end{itemize}
		Pliki ,,Backend-Start.bat'', ,,Backend-Stop.bat'' oraz ,,Backend-Start.sh'' i ,,Backend-Stop.sh'' służą do uruchomiania i zatrzymywania ,,backendu'' w wersji produkcyjnej,
		która pobierze i uruchomi najnowszą stabilną wersję ,,backendu'' aplikacji i bazę danych z serwisu ,,Docker Hub'', w przypadku pliku z przyrostkiem ,,DEV''
		zostanie uruchomiona kompilacja wersji deweloperskiej znajdującej się w lokalnym katalogu ,,Backend'' a następnie zostanie utworzony kontener z działającą aplikacją.
		W przypadku chęci zmiany ustawień ,,backendu'' aplikacji lub kontenera należy edytować - zachowując odpowiednie formatowanie plików yaml - plik w wersji produkcyjnej
		lub deweloperskiej:
		\begin{itemize}
			\item docker-compose-backend.yaml
			\item docker-compose-backend-dev.yaml		
		\end{itemize}
	\newpage	
	
	%Podsumowanie, wnioski
	\section{Podsumowanie}		
\end{document}